\documentclass{revtex4-1}

\usepackage[backref,breaklinks,colorlinks,citecolor=blue]{hyperref}  
\usepackage[all]{hypcap}
\usepackage{amsmath}
\usepackage{amssymb}
\usepackage{graphicx}
\usepackage{color}
\usepackage{natbib}
\usepackage{xspace}
\usepackage{nicefrac}
\usepackage[dvipsnames]{xcolor}
\usepackage[T1]{fontenc}
\usepackage{lmodern}
\usepackage{tablefootnote}
\usepackage{ulem}
\newcommand\myshade{85}
\colorlet{mycitecolor}{Turquoise}
\colorlet{mylinkcolor}{Turquoise}
\hypersetup{
 citecolor = mycitecolor!\myshade!black,
 linkcolor = mylinkcolor!\myshade!black,
 colorlinks = true,
}


\newcommand{\todo}[1]{\textcolor{red}{#1}}
\newcommand{\tocheck}[1]{\textcolor{green}{Check: #1}}
\newcommand{\citeme}{{\color{blue} CITE ME!}}
\newcommand{\ilya}[1]{\textcolor{magenta}{#1}}


\def\be{\begin{equation}}
\def\ee{\end{equation}}
\def\ba{\begin{eqnarray}}
\def\ea{\end{eqnarray}}

\begin{document}

\title{A 100 million year old thermometer}

\author{All author names to be added like this:\\}

\author{Ilya Mandel}
\email{imandel@star.sr.bham.ac.uk}
\affiliation{Institute of Gravitational Wave Astronomy and School of Physics and Astronomy, University of Birmingham, Edgbaston, Birmingham B15 2TT, United Kingdom\\ and\\
Monash Centre for Astrophysics, School of Physics and Astronomy, Monash University, Clayton, Victoria 3800, Australia}

\begin{abstract}

GDGTs in ocean-floor sediments provide a fossil record of ocean temperatures in the distant past.  Previous efforts to calibrate temperature prediction based on modern GDGT data have been suboptimal.  We apply modern machine-learning tools to the problem of calibrating a temperature predictor based on GDGT data.

\end{abstract}

\maketitle

\section{Introduction}


We wish to make the following three key points:

\begin{itemize}

\item Even the modern GDGT data set is fundamentally imperfect at being a predictor for temperature;

\item We discuss how best to predict temperature from GDGT observations using the modern data set for calibration, using all available data without pre-judgement;

\item We analyse the prospects for applingy tools calibrated on the modern data set to cretaceous and eocene data.

\end{itemize}


\section{Data}

Others should describe the data set.

\section{Temperature prediction: calibration and validation}

\section{Measuring temperature in the distant past}

\section{Discussion}

\begin{acknowledgements}
We thank the research councils for their munificence.

\end{acknowledgements}

\bibliographystyle{hapj}
\bibliography{paleo}

\end{document}


